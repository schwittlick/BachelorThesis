\chapter{Anforderungsanalyse}

\section{Anwendungsgebiet}
Die Arbeit besch�ftigt sich mit der Thematik anhand von Bildern und Bildsequenzen ( Videos ) aussagekr�ftige Informationen zu erhalten. Ein Bild repr�sentiert im Computer speicher ist nichts weiter, als eine
Matrix mit bestimmter H�he und Breite, welche multipliziert die Pixelanzahl des Bildes angeben. An jeder Position dieser Matrix sind die Farbwerte zu dem entsprechendem Pixel gespeichert. Durch die Betrachtung dieser Pixelmatrix kann unser menschliches Gehirn ohne weiteres dem Bild Informationen �ber dessen Inhalt entnehmen, folglich das Bild zu interpretieren. Die Interpretation eines Bildes ist ein komplexes Konstrukt und ist durch unsere Erfahrung als Mensch m�glich. Mehr zu diesem Thema in einem abgesonderten Block.
Ein Computer hat allerdings kein komplexes Gehirn und sieht ein Bild, was f�r einen Menschen z.B. gro�en emotionalen Wert haben kann schlichtweg als einen zweidimensionalen Array, welcher konkrete farbwerte enth�lt.
Anhand dieser repr�sentation des Bildes kann allerdings der Computer instruiert werden bestimmte Muster zu erkennen, oder bestimmte Operationen auf das Bild anzuwenden, wodurch letztendlich dem Bild Informationen 
entnommen werden kann. Diese Verarbeitung wird maschinelles Sehen oder Bildverstehen engl. Computer Vision genannt und besch�ftigt sich damit computergest�tzt Bilder auf eine menschliche Art und Weise zu interpretieren. Das Thema maschinelles Sehen wird im Abschnitt Maschinelles Sehen n�her beschrieben.

\section{Umsetzung}
\section{Anforderungen an den Prototypen}
Der Prototyp soll in der Lage sein unter m�glichst allen Wetterumst�nden Schiffe anhand des Videostreames der IP-Kamera zu erkennen. Dies soll nur tags�ber passieren.
Dieser Prototyp soll an unterschiedlichen Prten eingesetzt werden k�nnen, gegebenenfalls soll die Kamera an einen anderen Flussabschnitt bewegt werden k�nnen und die Software soll an die dort gegebenen lokalen Umst�nde anpassbar sein. Diese Anpassung soll anhand einer Grafischen Benutzeroberfl�che geschehen.
\subsection{Hardware}
Die Hardwarekomponente des Prototypen soll aus einer Kamera bestehen, die entweder Videosinal �ber das Netzwerk / Internet an den verarbeitenden Rechner inkl. Software sendet, oder direkt an einem Computer angeschlossen sein, welcher die Verarbeitung dann direkt vor Ort vornimmt. Allerdings ergibt sich dadurch das Problem, dass ein Computer am FLuss stehen mnuss, was nicht immer m�glich ist, da eine Stromzufuhr nicht immer gegeben sein kann. ( Wie soll denn �berhaupt eine Kamera dort funktionieren? Eventuell eine Kamera per Batterie laufen lassen mit Mobilem Hotspot?!! ) 
\subsubsection{IP-Kamera}
\subsubsection{Halterung}
\subsubsection{Benutzte Hardware}
\subsection{Software}
Die Softwarekomponente des Systemes soll inder Lage sein aus einer erh�hren Position am Flussrand Schiffe zu erkennen, zu verfolgen und gegebenenfalls klassifizieren und sonstige Informationen �ber das Schiff, wie z.B. Geschwindigkeit, L�nge aufzuzeichnen. Die Position aus der die Kamera auf den Fluss 'sieht' soll variabel sein und die Sofware soll an diese ver�nderbaren Umst�nde angepasst werden k�nnen. Diese Anpassung soll anhand einer grafischen Benutzeroberfl�che geschehen k�nnen.
\subsubsection{Benutzeroberfl�che}
Die grafische Benutzeroberfl�che soll es dem Benutzer erm�glichen die Algorithmen, die zur Erkennung und Verfolgung der Schiffe bnuetzt werden fein zu parametrisieren. Dazu sollen alle entscheidenden Parameter der Algorithmen aoffen gelegt werden und durch Standard Oberfk�chen-Elemente ver�nderbar sein.
\subsubsection{Performance}
\section{Abgrenzungskriterien}
\section{Methodik}
\subsection{Projektmanagement}
\subsection{Programmiermethodik}
Agile Programmierung
Extreme Programming
