\chapter{Einleitung}

In den folgenden Kapiteln wird beschrieben, wie das Problem dieser Bachelor Arbeit angegangen wurde.
\begin{enumerate}
\item \textbf{Einführung und Motivation}: Was ist die Problem- oder Aufgabenstellung und warum sollte man sich dafür interessieren?
\item \textbf{Analyse / Präzisierung des Themas}:Zunaechst wird die Aufgabe näher analysiert. Hier beschreibt man den aktuellen Stand der Technik oder Wissenschaft ("`State-Of-The-Art"'), zeigt bestehende
							Defizite oder offene Fragen auf und entwickelt daraus die Stoääörichtung der eigenen Arbeit.
\item \textbf{Definition}:Daraus ergibt sich eine Definition des Problemes und der naechsten Handlungsschritte.
\item \textbf{Grundlagen}:Dann werden die Grundlagen der Definierten Ziele beschrieben.
\item \textbf{Entwurf / Eigener Ansatz}:Anschliessend wird ein Entwurf des zu entwickelnden Systemes dargelegt.
\item \textbf{Implementierung / Prototyp}:Woraus eine tatsaechliche Implementierung resuliert.
\item \textbf{Test}:Welche getestet werden muss.
\item \textbf{Fazit / Zusammenfassung}:Und durch ein Fazit abgerundet wird.
\end{enumerate}