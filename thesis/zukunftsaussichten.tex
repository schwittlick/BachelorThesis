\chapter{Zukunftsaussichten}
F�r die Zukunft k�nnte ein System entwickelt werden, welches Schiffe sogar ausgesprochen schnell auf kosteng�nstiger Hardware erkannt werden k�nnte. Dieser Ansatzt spaltet das gesammte System in zwei Teile. Diese beiden Teile sind jeweils voneinander abgekoppelte Applikationen, wobei die erstApplikation im gro�en und ganzen der in dieser AStbeit beschriebenen gleicht. Der entscheidende Unterschied jodoch liegt darin, dass diese Sofware nicht unmittelbar daf�r benutzt wird, Schiffe zu erkennen und zu tracken, sondern lediglich daf�r benutzt wird einen Machine-Learning Algorithmus wie z.B. den Haar-Classifier zu trainieren. Dadurch kann die Software nach kurzer vorangegangener Parameterkalibierung eigenst�ndig Bilder von Schiffen erkennen und speichern, welche darauf zum Training des Haar-Klassifiers benutzt werden.
Der zweite Teil des Systemes w�re ein Computer mit Raspberry-Pi-�hnlicher Architektur und Leistung, welcher daf�r zust�ndig ist das digitale Videosignal ausschliesslich anhand eines Haar-Klassifiers zu analysieren, was selbst auf der heutzuztage billigsten Hardware in Echtzeit m�glich ist. (30euro RaspPi)