\chapter{Definition}
\section{OpenCV}
\subsection{Installation}
Da die Geschwindigkeit des Programmes von Anfang an wichtig ist wurde sich dazu entschieden so viel arbeitslast wie m�glich zu parallelisieren. Die Parallelsierung von Bildverarbeitsungsaufgaben bietet sich regelrecht an, da die Algorithmen wie folgt arbeiten: sie wenden die gleiche berechnung auf sehr viele zahlen/pixel an. Diese Arbeitsweise der Algorithmen �hnelt sehr stark der Arbeitsweise der GPU jeder Grafikkarte. 
Deswegen wurde sich dazu entschieden so viel wie m�glich der Arbeitslast auf die GPU auzulagern, da zu beginn der Bearbeitung vermutet wurde, dass eine Sequenzielle bearbeitung der Berechnungen nicht ausreichen wird, da die Software in Echzeit Videodaten verarbeiten k�nnen soll.
OpenCV unterst�tzt seit der version 2.4.6. die benutzung der CUDA Runtime API und unterst�tzt ausschlie�lich NVIDIA GPU's. Um die GPU f�r OpenCV zu benutzen muss OpenCV eigenst�ndig gebaut werden, die vorkompilierten versionen sind ohne CUDA support kompiliert. Daf�r muss CUDA , die neuesten Grafikkartentreiber und eine NVIDIA Grafikkarte vorhanden sein. 
Die Kompilierung hat sich als etwas kompliziert herausgestellt, sodass die neueste version von \url{http://www.github.com/itseez/opencv}  mit dem tag 2.4.7. benutzt werden musste um eine erfolgreiche kompilierung abzuschlie�en. Daf�r ist die Option WITH\_CUDA zu aktivieren, worauf man zus�tzlich das Verzeichnis der NVIDIA CUDA installation angeben muss. Weiterhin habe ich OpenGL f�r die beschleunigte videodarstellung und OpenMP f�r die parallelisierung auf den CPUs aktiviert, um die die Grundbedingungen f�r die Geschwindigkeit der Software ausreichend zu erf�llen.
\section{Qt}