\chapter{Kurzfassung}
In der Bachelorarbeit soll es darum gehen Schiffe bzw. Boote anhand eines Live-Videostreames zu erkennen und zu tracken. 

Zus�tzlich soll untersucht werden, inwiefern die Schiffe nach bestimmten Kriterien klassifiziert werden k�nnen, um gegebenenfalls Muster im Flussverkehr erkennen zu
k�nnen. M�gliche Klassifikatoren k�nnen der Schifftyp, die Gr��e des Schiffes und die Geschwindigkeit des Schiffes sein. 

Dadurch k�nnen detaillierte Statistiken �ber einen bestimmten Flussabschnitt erstellt werden, welche gegebenenfalls mit Statistiken anderer Flussabschnitte verglichen werden k�nnen. Deswegen soll die Hardwarekomponente des Systemes portabel und resistent gegen�ber Wettereinfl�ssen sein. Die Entwicklung der Hardware Komponente soll ebenfalls Teil der Arbeit sein und aus einer IP-Kamera inklusive 
Wetterresistenter Halterung bestehen. 

Es soll ein funktionierender Prototyp erarbeitet werden, der mit wenigen Schritten an einen anderen Ort umgezogen werden kann. Dies erfordert h�chstwarscheinlich eine Art
Kalibrierung des Systemes auf die ver�nderten Umgebungsfaktoren, wie Blickwinkel und Lichtverh�ltnisse des neuen Ortes, welche elementarer Teil des Systemes sein soll.

Die Erkennung, Verfolgung und Klassifizierung soll mittels bekannter und bereits implementierter Computer Vision Algorithmen geschehen und m�glichst robust gegen�ber eventuellen
Wettereinfl�ssen sein. Zur Klassifizierung der Schiffe soll die Methode des Maschinellen Lernens in Betracht gezogen werden, allerdings mit anderen Ans�tzen zur Klassifizierung verglichen werden.